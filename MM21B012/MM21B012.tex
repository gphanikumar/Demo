\section{MM21B012}

\title{Bernoulli's Equation}
\author{MM21B012}
\date{July, 2022}

\maketitle				

\subsection{Bernoulli's equation for in-compressible flow}

\begin{equation} 
\Psi = \frac{1}{2} \rho v^2 + \rho gz + p
\end{equation}

\begin{tabular}{|c|l|}
\hline
    Symbol & Description \\
\hline
    v & v is the flow speed at a point on the streamline \\
\hline
    g & g is the acceleration due to gravity \\
\hline
    z & z is the elevation of the point above a reference plane, (in the direction \\
    & opposite to acceleration due to gravity) \\
 \hline
    h & h is the pressure at the chosen point \\
\hline 
    $ \rho $ & $ \rho $  is the density of the fluid at all points of the fluid ( as the fluid \\
    & is in-compressible, it will be constant for all points of the the fluid ) \\
\hline
    $ \Psi $ & $ \Psi $ is a constant \\
\hline
\end{tabular}

\subsection{Explanation}
    In fluid dynamics, Bernoulli's principle states that an increase in the speed of a fluid occurs simultaneously with a decrease in static pressure or a decrease in the fluid's potential energy. The principle is named after Daniel Bernoulli who published it in his book Hydrodynamica in 1738. Although Bernoulli deduced that pressure decreases when the flow speed increases, it was Leonhard Euler in 1752 who derived Bernoulli's equation in its usual form. The principle is only applicable for isentropic flows: when the effects of irreversible processes (like turbulence) and non-adiabatic processes (e.g. heat radiation) are small and can be neglected.
    Bernoulli's principle can be derived from the principle of conservation of energy. This states that, in a steady flow, the sum of all forms of energy in a fluid along a streamline is the same at all points on that streamline. This requires that the sum of kinetic energy, potential energy and internal energy remains constant. Thus an increase in the speed of the fluid – implying an increase in its kinetic energy (dynamic pressure) – occurs with a simultaneous decrease in (the sum of) its potential energy (including the static pressure) and internal energy. If the fluid is flowing out of a reservoir, the sum of all forms of energy is the same on all streamlines because in a reservoir the energy per unit volume (the sum of pressure and gravitational potential $ \rho $gh) is the same everywhere.
