\section{ME21B060}
\subsection{what is a steady flow process}
A steady flow process is a process in which matter and energy flow in and out of an open system at same rates.
\subsection{Equation for steady Flow}
\begin{equation}`
 \dot Q-\dot W = \dot m *[(hi+vi^2/2+gzi)-( he+ve^2/2+gze)]
\end{equation}
\begin{tabular}{|c|r|l|}
\hline
sr.no & variable & meaning \\
\hline 
1 & $\dot Q$ & Rate of heat absorption by CV \\
2 & $\dot W$ & Rate of external work interaction \\
3 & $\dot m$ & Rate of mass entering into CV(kg/sec)\\
4 & zi & height above the reference line for the inlet.\\
5 & ze & Height above the reference line for an outlet.\\
6 & vi & velocity of entering steam.\\
7 & ve & velocity of leaving steam.\\
8 & hi & enthalpy of entering steam.\\
9 & he & enthalpy of outgoing stream. \\
\hline
\end{tabular}
\subsection{Explanation}
The steady flow energy equation (SFEE) is used for control volume system. To derive the equation few approximations are also taken into count. The incoming and outgoing mass rates are having the same value, Change in the total energy of a CV is assumed to be zero (0). In CV there is no mass accumulation due to mass flow. By assuming the inlet as state 1 and outlet as state 2 and by calculating the temperature and pressure at both states. We can find out the other thermodynamic properties of the CV such as work, heat transfer, enthalpy of incoming or outgoing flow.
