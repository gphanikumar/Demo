\documentclass{article}
\usepackage[margin=1in]{geometry}


\title{Equations of Motion}
\author{Amol Agrawal NA21B006}
\date{June 2022}

\begin{document}

\maketitle

\section{Basic equations of motion}
Suppose a body moving in a 2-D plane starts it motion with an initial velocity $\bf\vec u = u_x + u_y$ and moves with a constant acceleration of $\bf\vec a= a_x + a_y$. At any time $\bf t$ its velocity $\bf \vec v$ and displacement $\bf \vec S$ can be found using following equations:\\ 
\begin{equation}
    \vec v^2 = \vec u^2 + 2\vec a \vec S
\end{equation}
\begin{equation}
    \vec v = \vec u + \vec a  t
\end{equation}
\begin{equation}
    \vec S = \vec u t + 1/2 \vec a t^2
\end{equation}
\\
All of the above equations are derived using calculus and making appropriate assumptions that acceleration remains constant as $a_x$ and $a_y$ in the x and y direction respectively.
They are commonly known as "equations of motion" as they completely describe the motion of a rigid body.\\ \\
\begin{center}
\begin{tabular}{ |l|l| } 
 \hline
 $\vec u$ & Initial velocity possessed by the object which is resolution-ed into x and y components \\  
 $\vec a$ & Resolution-ed acceleration of the object which is defined as the rate of change of its velocity. \\ & For these equations of motion it is constant assuming that a constant force acts upon it.\\ 
 $\vec S$ & Displacement is defined as the separation between the initial and final positions of the object \\ & irrespective of the path that they followed. Unlike distance covered by the object, it is \\ & a vector quantity.\\ 
 t & Total time taken by the object to complete its journey\\
 \hline
\end{tabular}
\end{center}
\end{document}
