\section{CH21B101}

\documentclass{report}
\usepackage[utf8]{inputenc}
\usepackage{geometry}
 \geometry{
 a4paper,
 total={170mm,257mm},
 left=20mm,
 top=20mm,
 }

\begin{document}
\Large{\textbf{Roll NO. CH21B101}}
\newline
\newline
\huge{\textit{\underline{UNIVERSAL LAW OF GRAVITATION}}}
 
\section{Overview and Definition}
\Large{
\Large Universal law of Gravitation was coined by sir Issac Newton as an attractive force between two masses in a space.
\newline
\newline
\textbf{\textit{UNIVERSAL LAW OF GRAVITATION:}} \Large When considered two masses in a space with masses $m_1$ and $m_2$ respectively then the attractive force between them is directly proportional to the product of their masses [$m_1*m_2$] and inversely proportional to the square to the square of distance between them.
}
\section{Formulation}
 
\large{$F= G(m_1*m_2)/r^2$}
\huge{ 
\begin{center}
\begin{tabular}{ |c|c| } 
 \hline
 $G$ & Universal Gravitation Constant \\
 $r$ & Distance between $m_1$ and $m_2$ \\
 $m_1$ & Mass of body 1 \\
 $m_2$ & Mass of body 2 \\
 F & Force between the two bodies \\
 \hline
\end{tabular}
\end{center}
}
\section{Properties}
\begin{itemize}
\Large{
 \item It acts along the line joining the two masses.
 \item It is an attractive force with smaller magnitude compared to other forces like nuclear forces or electrostatic forces.
 \item It is an inverse square law which means the intensity of force inversely depends upon the square of the distance between the two masses.
} 
\end{itemize} 

\end{document}
