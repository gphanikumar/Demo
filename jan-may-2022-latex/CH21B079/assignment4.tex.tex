\documentclass{report}
\usepackage[utf8]{inputenc}
\usepackage{geometry}
 \geometry{
 a4paper,
 total={170mm,257mm},
 left=20mm,
 top=20mm,
 }

\begin{document}
\small{Roll NO:CH21B079}
\newline

\huge{\textit{COULOMB'S  LAW}}
 
\section{\underline{What is coulomb's law?}}
\Large{
\normalsizeThe The electrostatic force between two charges, as described by Coulomb's law.Coulombs law gives us the magnitude as well directions of force which will act between two charged particle separated by some distance in a space
\newline
\newline

\begin{document}
\huge{\textit{{ UNDERSTANDING OF COULOMB'S  LAW}}}
 
\section{\underline{Overview and Definition:-}}
\Large{
\normalsizeThe force exerted by stationary objects bearing electric charge on other stationary objects bearing electric charge, being repulsive if the objects have charges of the same sign, and attractive if the objects have charges of opposite signs. The strength of the force is described by Coulomb's law.
\newline
\newline
\textbf{\textit{DEFINATION OF COULOMBS'S LAW:}} \normalsize The magnitude of the electric force F is directly proportional to the amount of one electric charge, q1, multiplied by the other, q2, and inversely proportional to the square of the distance r between their centres. Expressed in the form of an equation, this relation, called Coulomb’s law, .
}
\section{\underline{Formula:-}}
 
\large{$F= k(Q_1*Q_2)/r^2$}
\huge{ 
\begin{center}
\begin{tabular}{ |c|c| } 
 \hline
 $k$ & permittivity of the vacuum \\
 $r$ & Distance between $Q_1$ and $Q_2$ \\
 $Q_1$ & Charge on particle 1 \\
 $Q_2$ & Charge on particle 2 \\
 F & Force between the two charged particles \\
 \hline
\end{tabular}
\end{center}
}

\end{document}

