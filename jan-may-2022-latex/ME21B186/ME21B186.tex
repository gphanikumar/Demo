\section{ME21B186}
Student shall edit this file and include stuff for the assignment

\documentclass{article}
\usepackage[letterpaper,top=2cm,bottom=2cm,left=3cm,right=3cm,marginparwidth=1.75cm]{geometry}
\usepackage{amsmath}
\usepackage{graphicx}

\begin{document}
\title{\textbf{\huge{Ampere's Law}}}
\author{ID2090 Assignment 4\\
Ronalyn Sequeira : ME21B186}
\maketitle

\begin{flushleft}
\large{Ampere's law can be written as the curl of B as :} \\
\end{flushleft}
\begin{center}
\boldmath
\large{$ $$\nabla$$ \times \vec{B} = {\mu_0}\vec{J}$} \\
\unboldmath
\end{center}

\large{\begin{tabular}{|c|c|}
\hline
 B  &  Magnetic field \\
 \hline
${\mu_0}$ & Permeability of free space\\
\hline
${\vec{J}}$ & Current density \\
\hline
\end{tabular}}\\
\\
\\
\large{The integral form of the Ampere's Law comes from the Stokes' theorem, which can be given as}
\begin{center}
\boldmath
\large{$$\oint{\vec{B}} \cdot d{\vec{l}} = {\mu_0}{\vec{I}_{enc}}$$}
\unboldmath
\end{center}

\begin{figure}[!ht]
    \centering
    \includegraphics[width=0.3\textwidth]{id_1.png}
    \caption{Current carrying loop}
    \label{fig:my_label}
\end{figure}

\large{$\oint{\vec{B}} \cdot d{\vec{l}}$ \text{is the line integral of the magnetic field around a closed current carrying loop.}
\large{$\vec{I}_{enc}$ \text{is the amount of current passing through the loop.}\\



\large{The equation helps in calculating the magnetic field around a current carrying closed loop.It even helps in calculating the current enclosed in a closed loop when the magnetic field is known.\\

The equation is known more for its application to symmetric systems.To find the magnetic field along a circular loop, around a cylinder carrying current axially, etc Ampere's law is very helpful and mak>
}

\end{document}
