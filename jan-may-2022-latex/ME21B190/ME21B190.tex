\section{ME21B190}
Student shall edit this file and include stuff for the assignment
\documentclass{article}
\usepackage[utf8]{inputenc}
\usepackage{graphicx}

\title{Assignment 4}
\author{Sheethal K }


\begin{document}

\maketitle
\section{Navier-stokes equation}
\begin{figure}[h]
\centering\includegraphics[width=0.5\textwidth]{air flow}
\end{figure}
\subsection {Continuity Equation:}\\[1.2pt]

$$\nabla.\vec{V}=0$$


\large\ The countinuity equation describes the trasport of some quantities like fluid or gas.The equation explains how a fluid conserves mass in its motion.\\[1.2pt]
\subsection {Momentum Equations:}\\[1.5pt]

$$\frac{d\vec{V}}{dt}=\frac{\partial V}{\partial t} + (V.\nabla)V$$\\[0.9pt]
\ Here the total derivative is the sum of change in velocity with time and the convective term\\[1pt]
$$\rho\frac{d\vec{V}}{dt}=-\nabla{p} + \rho\vec{g}+\mu\nabla^{2}\vec{V}$$\\[1pt]
 *The first term in RHS is the "Pressure gradient term" which tells in which direction the fluid flows\\[1pt]

*The second term is the "Body force term",which includes all type of forces\\[1pt]

*The third term "Diffusion term" 
\section{Explanation of variables}
\begin{center}
\begin{tabular}{ | m{5cm} | m{9cm} | }   
   \hline
   \textbf{Symbol} & \textbf{Explanation} \\
   \hline
   $$\vec{V}$$ & velocity vector \\
   \hline
   $\rho$ & density of fluid \\
   \hline
   $$\vec{g}$$ & acceleration due to gravity \\
   \hline
   $${p}$$ & pressure of fluid \\
   \hline
   $$\mu$$ & viscosity of fluid \\
   \hline
\end{tabular}
\end{center}
\end{document}
