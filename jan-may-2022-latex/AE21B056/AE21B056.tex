\begin{document}

\title{Beauty of Equations}
\author{Shreeya Padte}
\date{\today}
\maketitle

\section{Convolution Integral and Sum}

\subsection{Equation}
\subsubsection{Continuous time}
$$x(t) * h(t) = \int_{-{\infty}}^{\infty}{\text{x}(\tau) \text{ h}(t - \tau)  d\tau} $$
\subsubsection{Discrete time}
$$x[n] * h[n] = \sum_{k=-{\infty}}^{\infty} x[k] \hspace{2pt} h[n-k] $$

\subsection{Description}
\subsubsection{Continuous time}
It is the representation of a continuous-time linear time-invariant system in terms of its response to a unit impulse.
\subsubsection{Discrete time}
This corresponds to the representation of an arbitrary sequence as a linear combination of shifted unit impulses $\delta[n-k]$, where the weights in this linear combination are x[k].


\subsection{List of variables}
\begin{tabular}{l|l|l}
\hline
     & Variable & Use  \\
\hline
    1 & x(t) & Input signal which is continuous in nature \\
    2 & x[n] & Input signal which is discrete in nature \\
    3 & h(t) & Impulse response to x(t) \\
    4 & h[n] & Impulse response to x[n] \\
    5 & h[n-k] & Time shifted version of h[n] for computing the convolution sum. \\
    6 & x(t-$\tau$) & Time shifted version of x(t),for computing the convolution integral. \\
\hline
\end{tabular}

\end{document}
