\section{CE21B088}
\begin{center}
    \title{\textbf{\underline{{\huge Planck's Equation}}}}\\[\baselineskip]
    \date{July, 2022} \\[\baselineskip]
\end{center}
\maketitle


\textsf{{\large Planck's law describes the spectral density of electromagnetic radiation emitted by a black body in thermal equilibrium at a given temperature T, where there is no net flow of matter or energy between the body and its environment.}} \\[\baselineskip]
\textsf{{\large This formula is considered one of the most important physics formulas, as it is responsible for the birth of quantum mechanics, also television and solar cells.  Max Planck postulated in 1900, that energy was quantised and could be emitted or absorbed only in integral multiples of a small unit, which he called \textbf{energy quantum}.}}

{\LARGE \[\boxed{ E=hv }\]}
\textsf{\large Alternatively, this equation can be written as : }
{\LARGE \[\boxed{E=hc/\lambda}\]} \\
\textsf{\large This relation gives the energy of a photon E, known as \textbf{photon energy}. This relation states that the photon energy is \textbf{directly proportional to its frequency, \textit{v}.} }
\begin{table}[h!]
   \begin{center}
    \caption{Terms used}
        \begin{tabular}{| c | c |}
       
        \hline
            \textsf{\textbf{TERM}} & \textsf{\textbf{DESCRIPTION}}  \\\hline
            \textsf{E} & \textsf{Energy}  \\\hline
            \textsf{h} & \textsf{Planck's constant, whose value is $6.62607015$ × $10^{-34} m^2 kg / s$}  \\\hline
            \textsf{\textit{v}} & \textsf{Frequency of the incident light} \\ \hline
            \textsf{c} & \textsf{Speed of light, whose value is 299,792,458 m/s} \\ \hline
            $\lambda$ & \textsf{Wavelength of the incident light} \\ \hline
        \end{tabular}
    \end{center}
    \end{table}
    \\[2\baselineskip]
    \\[2\baselineskip]
    \begin{center}
        {\Large ---Thank You---}
    \end{center}

