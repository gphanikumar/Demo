\section{ME21B217}
\subsection{\centering{Bernoulli's Principle}}
\title{Bernoulli's Equation}
\author{V J Vivek ME21B217}		


\maketitle
\subsection{Description}
Bernoulli’s theorem, in fluid dynamics, relation among the pressure, velocity, and elevation in a moving fluid (liquid or gas), the compressibility and viscosity (internal friction) of which are negligible and the flow of which is steady, or laminar. First derived (1738) by the Swiss mathematician Daniel Bernoulli, the theorem states, in effect, that the total mechanical energy of the flowing fluid, comprising the energy associated with fluid pressure, the gravitational potential energy of elevation, and the kinetic energy of fluid motion, remains constant. Bernoulli’s theorem is the principle of energy conservation for ideal fluids in steady, or streamline, flow and is the basis for many engineering applications.
\subsection{Equation}
\begin{equation}
    P_1+\frac{1}{2}\rho v_1^2 + \rho gh_1 = P_2 + \frac{1}{2}\rho v_2^2 + \rho gh_2
\end{equation}
\begin{center}
\begin{tabular}{|c|c|}
    \hline
    \(P\) & Pressure of the fluid \\
    \hline
    \(\rho\) & Density of the fluid \\
    \hline
    \(v\) & Velocity of the fluid \\
    \hline
    \(h\) & elevation of the fluid \\
    \hline 
    \(g\) & acceleration due to gravitational force \\
    \hline
   
\end{tabular}
\end{center}


