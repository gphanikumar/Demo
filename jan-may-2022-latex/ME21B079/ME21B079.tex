\documentclass{article}

\usepackage[utf8]{inputenc}
\usepackage{graphicx}
\usepackage{array}
\title{Assignment 4}
\author{jayaram hemachander.}
\date{June 2022}

\begin{document}

\maketitle
\section{Maxwell Equation}
\begin{figure}[h]
\centering\includegraphics[width=0.5\textwidth]{maxwell}
\end{figure}
Faraday's law
$$ \frac{\partial\mathcal{D}}{\partial t} \quad  = \quad \nabla\times\mathcal{H}$$
Ampère's Law
$$ \frac{\partial\mathcal{B}}{\partial t} \quad  = \quad -\nabla\times\mathcal{E}$$
Gauss Law
$$ \nabla\cdot\mathcal{B}\quad  = \quad 0, \quad$$
Colomb's Law
$$ nabla\cdot\mathcal{D}\quad  = \quad \rho_{v} ,$$

\subsection{Faraday's Law}
When the magnetic flux linking a circuit changes, an electromotive force is induced in the circuit proportional to the rate of change of the flux linkage.
\subsection{Ampère's Law}
The magnetic field created by an electric current is proportional to the size of that electric current with a constant of proportionality equal to the permeability of free space
\subsection{Gauss Law}
Gauss's law for magnetism states that the magnetic flux B across any closed surface is zero
\subsection{Colomb's Law}
The closed line integral of magnetic field vector is always equal to the total amount of scalar electric field enclosed within the path of any shape
\section{Expansion of variables}



\begin{center}
\begin{tabular}{ | m{5cm} | m{5cm} | } 
  \hline
  \textbf{Symbol} & \textbf{Expansion}  \\ 
  \hline
  $$\mathcal{D}$$ & The volume of electric charge density \\ 
  \hline
  $$\mathcal{B}$$ & The magnetic field \\  
  \hline
  $$\mathcal{E}$$ & The electric field \\  
  \hline
  $$\mathcal{H}$$ & Magnetic field strength \\  
  \hline
  $$\rho_{v}$$ & Free Charge Density\\
  \hline
\end{tabular}
\end{center}
\end{document}
